%-------------------------
% Currículo em LaTeX
% Autor : Raphael Sena
% Baseado em: Sourabh Bajaj https://github.com/sb2nov/resume
% Licença : MIT
%------------------------

\documentclass[letterpaper,11pt]{article}

\usepackage{latexsym}
\usepackage[empty]{fullpage}
\usepackage{titlesec}
\usepackage{marvosym}
\usepackage[usenames,dvipsnames]{color}
\usepackage{verbatim}
\usepackage{enumitem}
\usepackage[hidelinks]{hyperref}
\usepackage{fancyhdr}
\usepackage[portuguese]{babel}
\usepackage{tabularx}
\usepackage{fontawesome}

\input{glyphtounicode}

\pagestyle{fancy}
\fancyhf{} % Limpa todos os campos de cabeçalho e rodapé
\fancyfoot{}
\renewcommand{\headrulewidth}{0pt}
\renewcommand{\footrulewidth}{0pt}

% Ajuste de margens
\addtolength{\oddsidemargin}{-0.5in}
\addtolength{\evensidemargin}{-0.5in}
\addtolength{\textwidth}{1in}
\addtolength{\topmargin}{-.5in}
\addtolength{\textheight}{1.0in}

\urlstyle{same}

\raggedbottom
\raggedright
\setlength{\tabcolsep}{0in}

% Formatação das seções
\titleformat{\section}{
  \vspace{-4pt}\scshape\raggedright\large
}{}{0em}{}[\color{black}\titlerule \vspace{-5pt}]

% Garante que o PDF gerado seja legível por máquina/compatível com ATS
\pdfgentounicode=1

%-------------------------
% Comandos personalizados
\newcommand{\resumeItem}[2]{
  \item\small{
    \textbf{#1}{: #2 \vspace{-2pt}}
  }
}

% Para um cabeçalho que não precisa estar em uma lista
\newcommand{\resumeHeading}[4]{
    \begin{tabular*}{0.99\textwidth}[t]{l@{\extracolsep{\fill}}r}
      \textbf{#1} & #2 \\
      \textit{\small#3} & \textit{\small #4} \\
    \end{tabular*}\vspace{-5pt}
}

\newcommand{\resumeSubheading}[4]{
  \vspace{-1pt}\item
    \begin{tabular*}{0.97\textwidth}[t]{l@{\extracolsep{\fill}}r}
      \textbf{#1} & #2 \\
      \textit{\small#3} & \textit{\small #4} \\
    \end{tabular*}\vspace{-5pt}
}

\newcommand{\resumeSubSubheading}[2]{
    \begin{tabular*}{0.97\textwidth}{l@{\extracolsep{\fill}}r}
      \textit{\small#1} & \textit{\small #2} \\
    \end{tabular*}\vspace{-5pt}
}

\newcommand{\resumeSubItem}[2]{\resumeItem{#1}{#2}\vspace{-4pt}}

\renewcommand{\labelitemii}{$\circ$}

\newcommand{\resumeSubHeadingListStart}{\begin{itemize}[leftmargin=*]}
\newcommand{\resumeSubHeadingListEnd}{\end{itemize}}
\newcommand{\resumeItemListStart}{\begin{itemize}}
\newcommand{\resumeItemListEnd}{\end{itemize}\vspace{-5pt}}

%-------------------------------------------
%%%%%%  CV COMEÇA AQUI  %%%%%%%%%%%%%%%%%%%%%%%%%%%%


\begin{document}

%----------CABEÇALHO-----------------
\begin{tabular*}{\textwidth}{l@{\extracolsep{\fill}}r}
  \textbf{\href{https://raphaelsena.com/}{\Large Raphael Sena}} & Email: \href{mailto:rsenares1@gmail.com}{rsenares1@gmail.com}\\
  \href{https://raphaelsena.com/}{raphaelsena.com} & Celular: \href{tel:+5531997167755}{+55 31 9 9716-7755} \\
\end{tabular*}

\section{Resumo}
    {Raphael Sena é um desenvolvedor de software que evoluiu do suporte de TI na prática para a construção e modernização de sistemas reais de mobilidade e ERP. Após uma experiência internacional inicial estudando em Sydney, iniciou sua carreira em funções de Service Desk/Help Desk, desenvolvendo uma base em troubleshooting, redes, hardware e suporte corporativo. Essa vivência técnica voltada ao cliente posteriormente o levou à engenharia de software. Hoje, atua como Engenheiro Fullstack na Modaxo (Empresa 1) em plataformas de bilhetagem eletrônica e gestão de receita do transporte em larga escala — entregando aplicativos móveis utilizados em diversas cidades brasileiras e contribuindo para esforços de modernização com Flutter e Java/Spring. Paralelamente ao trabalho profissional, desenvolve projetos pessoais, combinando mentalidade de produto com execução fullstack em mobile, backend, frontend e cloud/devops. Também é estudante de Engenharia de Software na PUC Minas (2023–2027), fortalecendo continuamente seus fundamentos de engenharia enquanto aplica esse conhecimento em ambientes de produção e projetos paralelos com \textbf{Java}, \textbf{Angular}, \textbf{PostgreSQL} e \textbf{Docker}}.


%-----------EXPERIÊNCIA-----------------
\section{Experiência}
  \resumeSubHeadingListStart

    \resumeSubheading
      {Modaxo - Empresa 1}{Belo Horizonte, MG - Brasil}
      {Engenheiro de Software}{Mai 2025 -- Atual}
      \resumeItemListStart
        \resumeItem{Projeto Sigom Cloud}
          {Sigom Cloud é um ERP/Sistema de Gestão de Receita do Transporte, oferecido no modelo SaaS. Atuei como desenvolvedor Fullstack com \textbf{Java} 1.8, \textbf{Spring Boot} e \textbf{AngularJS}. Otimizei a geração de relatórios de pagamento via PIX ao redesenhar o algoritmo e reduzir os caminhos de execução em \textbf{42,6\%}; para 1M de registros, reduzi o esforço estimado de \textbf{20M–1T para 1M} operações.}
        \resumeItem{Projeto SIGO}
          {SIGO é um aplicativo móvel de bilhetagem eletrônica para transporte público em cidades pequenas e grandes regiões metropolitanas do Brasil, como Guarulhos, SP; Florianópolis, SC; Uberlândia, MG e muitas outras. Atuei como desenvolvedor Fullstack utilizando \textbf{C\#} com \textbf{Xamarin} no app mobile e \textbf{Java} 1.7 com \textbf{Spring Boot} 1.5. O app está disponível na Apple App Store e na Google Play e \textbf{alcançou} mais de \textbf{177 mil pessoas}.}
        \resumeItem{Projeto SIGO 2.0}
          {O SIGO 2.0 é uma modernização do SIGO. Atuei como desenvolvedor Fullstack utilizando as tecnologias \textbf{Dart} com o kit \textbf{Flutter} para o aplicativo mobile e \textbf{Java} 1.7 e \textbf{Spring Boot} 1.5. Trabalhei nas seguintes funcionalidades: Extrato do Cartão, FAQ, Alertas, Cadastro de Usuário, Edição de Perfil, seções de Pedidos e Meus Cartões, além de integrações com o Moovit via GTFS e GTFS-RT para planejamento de rotas e acompanhamento de linhas em tempo real.}

      \resumeItemListEnd

%-------------------------------------------

   \resumeSubheading
      {AVASO Technology Solutions}{Belo Horizonte, Minas Gerais}
      {Field Support Engineer}{Set 2023 -- Atual}
    \resumeItemListStart
      \resumeItem{Service Desk (Freelance)}{
          Presto suporte em inglês para uma base de usuários multicultural. Diagnostiquei, pesquisei e resolvi problemas em desktops, notebooks, VMs, smartphones, servidores, sistemas de backup, dispositivos VoIP e periféricos. Realizei instalação e configuração de equipamentos e atualizei documentações em sistema de chamados para rastreio e relatórios.
        }
    \resumeItemListEnd

%-------------------------------------------

    \resumeSubheading
      {Sociedade Mineira de Cultura}{Belo Horizonte, Minas Gerais}
      {Técnico de TI}{Set 2021 -- Set 2023}
      \resumeItemListStart
        \resumeItem{Service Desk e Help Desk}{Configurei e verifiquei redes em desktops e notebooks. Verifiquei e instalei hardware. Realizei manutenção em computadores e notebooks. Prestei suporte técnico N1 utilizando Active Directory, sistema de chamados (CSC), VPN, plataformas de mapeamento de diretórios e impressoras, seguindo padrões de SLA estabelecidos. Colaborei em um projeto de mudança de telefonia para mais de 100 usuários, migrando redes lógicas de telefonia para VoIP e configurando/instalando equipamentos.}
    \resumeItemListEnd

  \resumeSubHeadingListEnd

  
%-----------FORMAÇÃO-----------------
\section{Formação}
  \resumeSubHeadingListStart
    \resumeSubheading
      {Pontifícia Universidade Católica de Minas Gerais}{Belo Horizonte, MG - Brasil}
      {Bacharelado em Engenharia de Software}{Jul 2023 -- Jul 2027}
        \resumeItemListStart
            \resumeItem{Campeão de Projeto Interdisciplinar}{Liderei uma equipe premiada em um projeto de software de alto impacto e aplicação no mundo real.}
            \resumeItem{Agência Experimental de Software - PMMG}{Como \textbf{Tech Lead}, lidero uma equipe para construir o escopo do software de processo seletivo de recursos humanos da Polícia Militar de Minas Gerais, impactando mais de \textbf{40.000 candidatos} por concurso público.}
        \resumeItemListEnd
    \resumeSubheading
      {Kogarah High School}{Sydney, NSW - Austrália}
      {Ensino Médio}{Jul 2017 -- Dez 2017}
  \resumeSubHeadingListEnd


%-----------PROJETOS-----------------
\section{Projetos}
  \resumeSubHeadingListStart
    \resumeSubItem{\href{https://raphaelsena.com}{raphaelsena.com}}
      {Portfólio pessoal me apresentando e destacando meus trabalhos.}
      \resumeSubSubheading{Typescript, Next.js, React, TailwindCSS}{\hfill
      {\href{https://raphaelsena.com}{\faLink}}
      {\href{https://github.com/raphael-sena/portfolio}{\faGithub}}}
      
    \resumeSubItem{\href{https://ongremediar.com.br/}{Remediar}}
      {Plataforma ERP premiada desenvolvida para a ONG Remediar, uma organização de caridade voltada à doação de medicamentos.}
      \resumeSubSubheading{Java, Spring Boot, PostgreSQL, Docker, Typescript, Next.js, React, TailwindCSS}{\hfill
      {\href{https://ongremediar.com.br/}{\faLink}}
      {\href{https://github.com/raphael-sena/remediar}{\faGithub}}}
      
    \resumeSubItem{DressManager}
      {Aplicação web para gestão de uma loja de vestidos.}
      \resumeSubSubheading{Java, Spring Boot, PostgreSQL, Docker, Typescript, Next.js, React, TailwindCSS}{\hfill
      {\href{https://github.com/raphael-sena/dress-manager}{\faGithub}}}
      
  \resumeSubHeadingListEnd


%-----------COMPETÊNCIAS-----------------
\section{Competências}
  \resumeSubHeadingListStart
    \resumeSubItem{Linguagens de Programação}
        {Java, Typescript, C\#, Node.js, Dart, SQL}
    \resumeSubItem{Frontend}
        {HTML, CSS, React, Next.js, Angular, TailwindCSS}
    \resumeSubItem{Mobile \& Cross-platform}
        {Flutter, Xamarin}
    \resumeSubItem{Backend \& Frameworks}
        {Spring Boot, Node.js}
    \resumeSubItem{Bancos de Dados}
        {Oracle, PostgreSQL, MySQL, SQLite}
    \resumeSubItem{DevOps \& Cloud}
        {Docker, Cloudflare, Linux, Github Actions}
    \resumeSubItem{Observabilidade}
        {Zipkin, Grafana e Prometheus}
    \resumeSubItem{Mensageria \& Cache}
        {RabbitMQ, Redis}
    
 \resumeSubHeadingListEnd

 %-----------IDIOMAS-----------------
\section{Idiomas}
    \resumeSubHeadingListStart
      \item{
        \textbf{Inglês} (C1) \textbar{}
        \textbf{Alemão} (B1) \textbar{}
        \textbf{Espanhol} (A2) \textbar{}
        \textbf{Português} (Nativo)
      }
    \resumeSubHeadingListEnd


%-------------------------------------------
\end{document}
